\documentclass{article}
\usepackage[margin=0.8in]{geometry}

\title{Procedure for Reading This Set of Books}
\author{Donald Knuth}
\date{18th Feb, 2023}

\begin{document}
\maketitle
\newpage

\tableofcontents
\newpage

\section{Preface}
% \setcounter{page}{0}
The process of preparing programs for a digital computer is especially attractive, not only because it is ecomically and scientifically rewarding, but also because it can be an aesthetic experience muct like composing poetry or music. The chapters in book are not meant to serve as an introduction to computer programming, the reader is supsposed to have had some previous experience. The reader should possess:
\begin{enumerate}
    \item Some idea of how a stored-program digital computer works(manner in which instructions can be kept in the machine's memory and successively executed.)
    \item An ability to put the solution to problems into such explicit terms that a computer can "understand" them.
    \item Some knowledge of most elementary computer techniques, such as looping, the use of subroutines and the use of indexed variables.
    \item A little knowledge of common computer jargon--"memory", "registers", "bits", "floating point", "overflow", "software".
\end{enumerate}

This set of books is intended for people who will be more than just casually interested in computers, yet it is by no means only for the computer specialist. The main goal has been to make these programming techniques more accessible to the many people working in other fields who can make fruitful use of computers, yet who cannot afford the time to locate all the necessary information that is buried in technical journals. Knuths approach has been to try and distill the vast literature by studyint the techniques that are most basic, in the sencse that they can be applied to many type sof programming situations. He in this series has attempted to coordinate the ideas into more or less of a "theory", as well as to show hoew the theory applies to a wide variety of practical problems.

"Non numetical analysis" is a terribly negative name for this field of study, and "information processing" is too broad a designation for the materials Knuth has considered, as well as "programming techniques" is too narrow. Therefore Knuth names the subject matter covered in this book as "analysis of algorithm".  This name is meant to imply "the theory of the properties of particaular computer algorithms."
\newpage

\section{General Outline}
The complete set of books, entitled The Art of Computer Programming, has the following general outline:
\begin{itemize}
    \item[] Volume 1. Fundamental Algorithms 
        \begin{itemize}
            \item[] Chapter 1. Basic Concepts
            \item[] Chapter 2. Information Structures
        \end{itemize}
    \item[] Volume 2. Seminumerical Algorithms 
        \begin{itemize}
            \item[] Chapter 3. Random Numbers
            \item[] Chapter 4. Arithmetic
        \end{itemize}    
    \item[] Volume 3. Sorting and Searching
        \begin{itemize}
            \item[] Chapter 5. Sorting
            \item[] Chapter 6. Searching
        \end{itemize}    
    \item[] Volume 4. Combinatorial Algorithms 
        \begin{itemize}
            \item[] Chapter 7. Combinatorial Searching
            \item[] Chapter 8. Recursion
        \end{itemize}
    \item[] Volume 5. Syntactical Algorithms
        \begin{itemize}
            \item[] Chapter 9. Lexical Scanning
            \item[] Chapter 10. Parsing
        \end{itemize}
\end{itemize}

Volume 4 deals with such a large topic, it actually represents three separate books ( Volumes 4A, 4B and 4C). Two additional volumes on more specialized topics are also planned: Volume 6, \textit{The Theory of Languages}; Volume 7, \textit{Compilers}.



\end{document}